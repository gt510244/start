\documentclass[dvipdfmx]{jsarticle}
\usepackage{otf,amsmath,amssymb}
\usepackage[dvips]{graphicx}
\usepackage[english]{babel}
\usepackage{natbib}
\usepackage{lscape,bm}
\title{Municipal Merger and Strategic Behavior in South African Municipality}
\author{Tsuyoshi Goto\footnote{Graduate School of Economics, Osaka University, Japan.  Address: 1-7 Machikaneyama, Toyonaka, Osaka 5600043, Japan. Email: sge010gt@student.econ.osaka-u.ac.jp}, Sandra Sekgetle\footnote{National Treasury, South Africa.}, Takashi Kuramoto\footnote{Konan University.}}

\begin{document}
\maketitle
\newtheorem{As}{Assumption}
\newtheorem{Hy}{Hypothesis}
\newtheorem{Lem}{Lemma}
\newtheorem{Prop}{Proposition}
\newtheorem{Def}{Definition}
\newtheorem{Th}{Theorem}
\newtheorem{Cor}{Corollary}
\begin{abstract}
Employing South African data, this paper investigates whether the free-ride behaviour of municipalities before the municipal demarcation changes happen. With developed countries' data, many research show that there are opportunistic behaviours of municipalities such as over-issuance of debt and over-expenditure before their mergers since the burden of them will be shared by newly constructed municipalities.\\
\quad However, interestingly and surprisingly, we show that South African municipalities did not increase the amount of expenditure and borrowings before their mergers and even decrease the amount of borrowings then. It is considered that this result is caused by a strict monitoring system between municipalities, a short preparation term for amalgamations, and a low capacity to issue a debt. This paper shed the light to utilize the data of developing countries and is the first paper to show there were reductions of borrowings before municipal mergers.

\end{abstract}
\textbf{Keywords}: Municipal merger; Common pool problem; Free ride\\
\textbf{JEL Classification}: H72; H77; H83
\section{Introduction}
\quad As local governments or municipalities have faced many problems such as urbanization, a financial viability and a divergence of needs for public services, various reforms have been tried for local governments all over the world. In particular, many countries such as Australia, Belgium, Canada, Denmark, Finland, Germany, Japan, and Sweden have implemented municipal mergers to deal with the problems. Since there are a lot of data and research topics related to municipal mergers and demarcation changes of local governments, many researches about mergers in developed countries have done.\\
\quad Among the papers, a topic which various studies are concerned is about a fiscal common pool problem. The fiscal common pool problem is a closely tied concept to free-ride problem and is formalized by \cite{Weingast1981}. The basic idea of the fiscal common pool problem is that, if $n$ municipalities with debt get merged, a debt burden of each municipality will be shared among the merged municipalities and will be $1/n$ of original repayment cost. Employing this idea and the data of different countries, various papers are written \citep{Allers2016, Blom-Hansen2010, Frid2015, Hansen2014, Hinnerich2009, Hirota2017a, Jordahl2010, Nakazawa2016}. In this strand, most papers show that there is the fiscal common pool problem and some papers show that the extent of free-ride behaviour of municipalities depends on the scale of each municipality. Therefore, the empirical results successfully verify the existence and mechanism of the fiscal common pool problem as \cite{Weingast1981} show.\\
\quad However, there may be several shortfalls for the literature. One possible point is the awareness of each municipality or government. Why do not municipalities realize the others issue a lot of debts and why do not they stop it if they notice it? If the financial information of municipalities to be merged is transparent and there is a preventive regulation, the fiscal common pool problem may not be happen. \cite{Milesi-Ferretti2003} theoretically shows that the transparency reduce the amount of unnecessarily over-issued debt and the optimal regulation will require a fiscal surplus if the welfare will suffer from the fiscal deficit created by a myopic government, where the required fiscal surplus is the increasing function of the probability about the existence of the myopic government. This fact implies that a rational central government will implement the strict regulation for local governments suspected to be myopic if the central government can tell which is myopic. A regulation for a debt issuance is partly researched and shown that it works to stop over-issuing. \cite{Nakazawa2016} shows that Japanese municipalities with the regulation for the debt issuance did not issue more debt than ones without it. Considering this, the extent of debt issuance is affected by the regulation and the transparency. \\ 
\quad Another point is that the previous papers focus only on developed countries. If we deal with other types of countries, the result of research may be varied. The problem such as urbanization and a financial sustainability is a serious problem in developing countries now. Potentially, more and more developing countries will implement demarcation changes of local governments and they will refer to precedents of municipal mergers though the case of developed countries is not able to apply simply.\\
\quad Employing South African data, this paper tries to fill these shortfalls. We utilize the panel data of South African local governments, called municipalities. South Africa is one of the advanced African countries, where the local public finance and governance system is maintained and a lot of data is also available. Indeed, South Africa has a good budget transparency and is listed as the best country in Open Budget Index Rankings \citep{IBP2017}. For the municipal mergers, merged municipalities were monitored using their data and regulated to stop making new contracts. These facts may affect the result of the fiscal common pool problem. In addition, South Africa is one of the developing countries as well, though it is sometimes classified as a semi-developed country and experienced some problems along with their development. Many developing countries will face the problems that South Africa already experienced, including the problems entailed by municipal mergers.\\
\quad This paper focuses on the free-ride behaviour of municipalities because of the fiscal common pool problem before the municipal demarcation changes happen, as many papers did, using South African data. Although many research shows there are opportunistic behaviours of municipalities, interestingly and surprisingly, we show that pre-merger municipalities did not increase and even decrease the amount of borrowing per capita in South Africa.\\
\quad We focus on the South African municipal demarcation change implemented in August 2016. For research, there are also several advantageous points to investigate the South African situation. Firstly, in South Africa, municipal mergers are determined by an independent governmental board, called “Municipal Demarcation Board (MDB)” in every five years, not municipalities by their own. If municipalities decide mergers by themselves, the random assignment is failed and their observed opportunistic behaviours may have an endogeneity. However, the treatment, the determination of mergers, was considered to be almost randomly and sometimes suddenly assigned by MDB. This enable us to make a quasi-experimental setting for the analysis. Secondly, since the demarcation change of each municipality was implemented in the same time, we can easily capture the behavioural changes by mergers employing the Difference-In-Difference (DID) method. Some previous papers use annual data where each merging municipality merges in different timings. This may cause biased weight setting for treatment period: for example, when municipality A get merged on January 1 of 2017 and municipality B get merged on December 31, can a year dummy variable of 2017 for capturing treatment effect be the same one? In this paper, the timing of mergers is the same among all merged municipalities and it is ideal for the DID method. Thirdly, we can obtain several time-span data of municipalities such as quarterly, half-yearly, and annual data in South Africa. Since the announcement of mergers' determination was delivered from December 2015 to January 2016 to municipalities, annual data may fail to capture the behavioural changes after the announcement, but we mainly use half-yearly data and can confirm it using other time-spanned data to check the robustness. \\
\quad Utilizing these good features of the South African situation, we analyzed the fiscal common pool problem and did interviews for some municipalities and institutions related to a municipal demarcation change\footnote{We had interviews with MDB, the National Treasury, the Department of Cooperative Governance and Traditional Affairs (CoGTA), South African Local Government Association (SALGA), Mbombela Local Municipality, Polokwane Local Municipality, JB Marks Municipality, and Rand West Municipality.}. The analysis investigates the amount of borrowing per capita, namely debt issuance per capita  In addition, throughout the interviews, we could obtain some supportive information for the results of the regressions.\\
\quad This paper consists of six parts. The next section explains the mergers in 2016 and the institutional background. Section 3 describes our hypotheses and the identification strategy. Section 4 outlines the data and summary statistics. Section 5 reports the regression results and we also discuss the implication of the results there. Section 6 concludes.
\section{Institutional background}
\quad South Africa earned its democratic status in April 1994. The country is administratively governed via a three tier system consisting of national government, provincial government and local government. The Constitution of South Africa (1996) refers to each government as a sphere, thereby allowing space and some level of distinctiveness. At the same time, section 40(1) of the Constitution provides clarity on interrelations and interdependence of the spheres of government. \\
\quad Section 156 of the Constitution allocates ‘original’ powers and functions to municipalities by listing ‘local government matters’ over which local government has authority. Additional powers and functions can be transferred by national and provincial governments to local governments as a sphere, or to individual municipalities in terms of section 156(2).\\ 
\quad The post-apartheid South Africa has experienced strategic institutional restructuring mainly affecting local government. In an attempt to rationalize local government, South African government has progressively and systematically reduced the number of municipalities from 1262 in the 1995/96 financial year down to 257 after the 2016 local government elections.  On the other hand the number of provinces remained unchanged as nine provinces which is the number that was originally agreed to at the dawn of the democratic South Africa. Provinces share different numbers of municipalities some of which includes metropolitan municipalities and other do have metropolitan municipalities within their provincial boundaries. The Municipal Demarcation Board (MDB) is an independent authority responsible for the determination of municipal boundaries in South Africa. This is an independent institution established in terms of the Constitution of the Republic of South Africa. Before the formation of this national board, provinces were determining boundaries through demarcation board within their jurisdictions. The Municipal Structures Act (No. 117 of 1998) covers the delimitation of wards for local elections and the capacity assessments of municipalities. The activities of the MDB are also in accordance with other pieces of legislation, such as the Public Finance Management Act (No. 29 of 1999) and the Municipal Finance Management Act (No. 56 of 2003).\\
\quad Several reasons have been cited for the consolidation of municipalities. However in 2015 the Department of Cooperative Governance and Traditional Affairs (CoGTA) developed functionality and viability framework for all municipalities. The analysis concluded that roughly one-third of municipalities are dysfunctional and not viable. This means that most of these municipalities classified as dysfunctional and non-viable are not able to meet their constitutional mandates (to deliver required basic services). This framework was used to inform the 2016 municipal mergers as a mean of improving the affected municipality’s functionality and viability. \\
\quad Post the new demarcations, still not all the problems have been resolved. Some municipalities still face financial stress and challenges associated with viability and municipal integration. This is just an indication that mergers will not solve all problems at once but in due course and with appropriate systems being institutionalized.\\
\quad Chapter 7 of the Constitution makes provision for three categories of municipalities; metropolitan municipalities, (Category A), local municipalities (Category B) and district municipalities (Category C). Metropolitan municipalities exist in the eight biggest cities and collectively accounts for the largest population size which is estimated at 21 million people. These metropolitan municipalities accounts for about 80 percent of the country’s gross domestic product. Local municipalities are areas that fall outside of the eight metropolitan municipalities with 31 million population. There are 205 local municipalities. District municipalities are made up of a number of local municipalities that falls within one broader region and are called district municipalities. Each district is made up of about 4 to 6 local municipalities. The district municipality is responsible for the co-ordination of development and service delivery in the whole district and has its own administration [staff]. The municipalities within a district municipality are differentiated into large urban towns, small towns and rural municipalities, all of which are referred to as local municipalities and have their own local council.\\
\quad Municipalities in South Africa are primarily responsible for services such as water, sanitation, municipal roads, refuse removal, electricity reticulation, environmental health services and town planning. Furthermore, they develop and maintain parks, recreational facilities, local markets and local transport facilities. For these services, municipalities are often allocated funding through the process of the division of the collected revenue. In addition to these constitutionally guaranteed functions and depending on the capacity of the municipalities (mainly metropolitan municipalities), they often perform other functions including housing delivery, primary health care and community services such as libraries and museums. Depending on whether the municipality is a local or a district, there are assigned different roles and responsibilities (functions). For example, some municipalities are given a function of water service provider (WSP) because they have capacity to deliver that service. Providing all the services means that the three spheres of government have to work together, as each sphere is responsible for its own functions as well as for those that overlap between the spheres. For this reason, South Africa has adopted a system of cooperative governance which is enshrined in Chapter 3 of the constitution. This requires municipalities and provincial and national departments to work together in a cooperative manner.  \\
\quad The Constitution requires that local government must be an autonomous and financially self-sufficient sphere of government, which is also responsible for creating its own economic development path. The main revenue sources for municipalities are charges generated from service charges, property rates and transfers from the national government. The greatest source of own revenues is service charges which municipalities collect for providing water, sanitation, electricity and refuse removal to households and businesses. However, municipalities cover the cost of services provided to poor households with transfers from government (cost of free basic services). Across all municipalities, transfers account for about a quarter of revenues and own revenues for three quarters. However, in the case of poorer municipalities, transfers account for the majority of revenues. There are two forms of transfers, conditional (for a specific purpose) and unconditional which is for general use. Conditional grants are mainly used for capital spending and unconditional grants for operational expenditure.
\section{Hypotheses and identification strategy}
\quad The main aim of this paper is to investigate whether there is a free-ride behaviour caused by the fiscal common pool problem along with municipal mergers. As many papers show, municipalities may increase their debt and expenditure just before their mergers after receiving the announcement of their mergers. Thus, the first hypothesis is below.
\begin{Hy}
Municipalities increase their debt after receiving the announcement of their mergers.
\end{Hy}
\quad In addition, some papers such as \cite{Hinnerich2009} suggests that the extent of free-ride behaviour depends on the relative size of municipalities to be merged. Thus, the second hypothesis is the following.
\begin{Hy}
Municipalities increase their debt depending on their relative size in a newly constructed municipality.
\end{Hy}
\quad These two hypotheses are very basic ones and can be verified by the Difference-In-Difference (DID) strategy. To verify the two hypotheses, we use two treatment variables. The first one is just a treatment dummy for merged municipalities. This dummy means whether municipalities will get merged or not. The second one is the index of free-ride made by Gross Value Added (GVA) data. GVA corresponds to Gross Domestic Product and is calculated for each municipality. Although other papers use population for calculating the index of free-ride, GVA captures the economic activity and can more precisely reflect what extent a municipality depends on the counterpart municipalities. We can regard the index as to what extent each municipality could enjoy the free-ride behaviour as \cite{Hinnerich2009} has already used.
\begin{center}Figure 1:Three types of municipal demarcation\\
\includegraphics[width=10cm,clip]{1.png}
\end{center}
\quad The way how we made the index of free-ride depends on types of mergers. In the 2016 demarcation, there were 21 mergers and 49 old municipalities get merged into 28 new municipalities. There were three types of mergers as Figure 1 shows. 1)The first one was that municipalities simply got merged into one municipality. This type of merger did not entail the split of any municipality. Consider municipalities $A$ and $B$ get merged here and $n_i$ shows the GVA of a municipality $i$. For calculation, we employ GVA in the first half year of FY2014/15. It is very simple to calculate the index of free-ride since municipality $A$’s and $B$'s index of free-ride will be $1-\frac{n_A}{n_{A}+n_{B}}$ and $1-\frac{n_B}{n_{A}+n_{B}}$ for a newly formed municipality $A+B$. 17 examples are in line of this merger type. 2)The second one entailed the split of a municipality and the divided parts absorbed into other existing municipalities. Consider municipalities C will be divided and its split parts get merged into A and B. In this case, $C$'s index of free-ride is calculated as $1-\frac{n_{C}}{n_{A}+n_{B}+n_{C}}$ and $A (B)$'s index of free-ride is $1-\frac{n_{A}}{n_{A}+\frac{n_{C}}2}$ ($1-\frac{n_{B}}{n_{B}+\frac{n_{C}}2}$). This index of free-ride is actually approximated since we cannot know what extent one part of municipal C is merged into A and the other part goes into B. 3)The third one entailed the split of municipalities and the parts of them were not absorbed but did form a newly municipality. Only one example can be applied in this case and we cannot calculate the index of free-ride in this case since existing municipalities can leave a newly formed municipality with their burden. Even if this is not true, since there is only one example here, we omit this sample.\\
\quad The decisions for the demarcation were made not by municipalities but by the Municipal Demarcation Board (MDB) and MDB announced their decisions. Although municipalities had several chances to submit their opinions about the demarcation to MDB, basically MDB made decisions depending on the section 24 and 25 of the Municipal Demarcation Act (No.27 of 1998), where 16 criteria were written. According to MDB, there is no crucial criterion, but they consider them comprehensively. This means that the treatment may be assigned randomly and there is little room for the endogeneity. In addition, although the Department of Cooperative Governance and Traditional Affairs (CoGTA) intended to make municipalities better through the municipal demarcation using the analysis based on the functionality and viability framework, the decisions for demarcation were made by MDB independently and this intention of CoGTA was said to fail since the functionality and viability of municipalities did not improved through the demarcation \citep{Ncube2017}. These facts may show that the treatment of mergers is fairly exogenous. However, in the interviews, several people suggest that the decision of municipal demarcation is affected by the local politics. When a ruling party, which is typically African National Congress (ANC), is weak in a municipality, they may survive in the next election if the municipality get merged with a municipality where the same ruling party is very strong. Since the Municipal Demarcation Act prescribes that anyone can submit their opinion about mergers to MDB, some people doubt that local politicians and the groups cooperating with them submit their request for mergers. If this is true, it may affect the exogeneity in our regression. Thus, we employ a control variable, which captures ANC's occupancy rate in a local parliament.\\
\quad The municipal demarcation change in 2016 was implemented on 3 August simultaneously among the all announced municipalities. The final decision for the delimitation of the ward, which is the basic component of municipalities' demarcation, was decided in December 2015 (MDB, 2017) and was announced to each municipality after that. From several interviews, we confirm that although some municipalities was said that they had expected the mergers, it was a sudden announcement for most municipalities and was an unexpected content for even municipalities which had expected the mergers since the formal combinations of municipalities were firstly announced in December 2015 and some municipalities noticed it in January 2016 for sure. Thus, the time variable which captures the treatment effect should be introduced from January 2016, or from December 2015 at least. Since we mainly employ half-yearly data, the time dummy for the second half year of FY2015/16 will capture this effect.
\section{Data and summary statistics}
What is important for the DID strategy is the assumption of parallel trend. Under this assumption, the trends of treatment group and control group are parallel until treatment is not assigned and there is no big difference between the groups.\\
\quad Considering this condition, we mainly employ the half-yearly panel data from FY2011/12 to FY2015/16. The timing of demarcation comes every five years and we omit the data before FY2011/12 since there may be a structural change. Year data and Quarter data were also available and used for a robustness check. The data set contains the data of 232 municipalities although the number of all municipalities are 234 and we omit two municipalities from them because the way of their merger was special as explained in section 3\footnote{This way of merger is shown as `3) Sprit and No absorption' in the Figure 1 and corresponds to 3)the third type of merger in the section 3.}. In the data, 47 municipalities got merged and the other 185 municipalities were remained.\\
\quad The dependent variable in this paper is borrowing per capita by municipalities. This is not stock index but flow index. In particular, the borrowing shows that the amount of capital expenditure funded by external borrowings and the external borrowings are not utilized for the other expenditure, namely operational expenditure.  The borrowing accounts 17.5\% of the capital expenditure. The amount of borrowing tend to be zero and vary in time and among municipalities. This means that municipalities do not raise their revenue from the borrowing very much, and checking through a histogram of the borrowing, the data of the borrowing may be considered to be censored at zero. Therefore, we also employ a tobit model for a analysis of the borrowing.\\
\quad The mean trends of the borrowing and the borrowing per capita are shown in Figure 2 and Figure 3, respectively. At a glance, Figure 2 show that the parallel trend assumption probably holds though Figure 3 does not. From both graphs, we can confirm that the borrowing (or borrowing per capita) of treatment group seems to reduce after the first half year of FY2015/16.\\
\quad The summary statistics are summarized by treatment and control groups in Table 1. T-statistics show whether the differences of mean value between treatment and control group is significant. We can confirm that the differences of the borrowing between treatment and control are not significant in their mean, while the differences of the borrowing per capita are significantly different.\\
\quad These figure and table may show that the parallel trend assumption for the borrowing per capita is doubtful. However, the differences of it between treatment and control group may be captured by a treatment dummy that we introduce later. In addition, the result of pseudo test is $F(8,2292) =1.41$ and it shows the parallel trend may be satisfied. Thus, we can consider that the parallel trend assumption holds here. 
\begin{center}Figure 2:Trend of borrowing\\
\includegraphics[width=13cm,clip]{2.eps}
\end{center}
\begin{center}Figure 3:Trend of borrowing per capita\\
\includegraphics[width=13cm,clip]{4.eps}
\end{center}
\begin{center}Table 1:Summary statistics by groups\\
\scriptsize\begin{tabular}{c|rrrrrr|c} \hline
Variables &Control Mean&Control Std Err.  & \#Control &Treated Mean &Treated Std Err. &\#Treated &t-value \\ \hline
Population & 229844.9 &12758.62&  1850 & 209201.6  &23878.74 & 470 &0.7365\\ 
Gross Value Added  &11806.33&958.9898 &1850&10620.3  & 1609.517 &470 &  0.5734\\
Area&5434.83&129.89&1850&4259.93&216.34&470 &4.1984***\\
ANC seat &0.6438&0.0041 &1850 & 0.6676&0.0070 &470 &-2.6823***\\ 
Metro&0.0324&0.0041&1850  &0.0425 &0.0093&470 &-1.0736\\
Secondary city&0.0810&0.0063&1850&0.0851&0.1288&470  &-0.2841\\
Borrowing&1.64e+07 &2940568  &1850 &9524138 &3597770&1850 &1.1236\\
Borrowing per capita&19.347& 1.4113&1850&11.206 &1.7734&470  &2.7694***\\ \hline
\end{tabular}\normalsize
\end{center}
\quad In the regression, six control variables are added. First one is population. This is available not half-yearly, but in every five years. Therefore, it is constructed by a linear interpolation using the data of 2011 and 2016. Second one is Gross Value Added (GVA), which is calculated by municipalities and corresponds to Gross Domestic Product. This is based on millions of South African Rand at 2010 prices. Third and Fourth variables are area and ANC seat. ANC seat shows the occupancy rate of ANC in a municipal parliament. ANC is a ruling party in the most of municipalities and in the national parliament. South Africa has local election every five years and this data is based on the result of 2011 local election. Metro and Secondary city is the fifth and sixth variables. In South Africa, large cities are called as metropolitan municipalities or 'Metro'. Their role and structure is expected to be more comprehensive than the other municipalities and is distinct from the others in laws including the constitution. Secondary city is a category of municipalities, which is not prescribed in law, and relatively large cities are listed as secondary cities.\\
\quad Summary statistics of control variables are listed in Table 1. The column of t-value shows that the means of Area and ANC seat is different between treatment and control groups. This means that we should consider these control variables in the regression. Actually, four control variables including Area and ANC seat take constant values in this data set and they are omitted in fixed effect model. Since the fixed effect model captures a heterogeneity among each municipalities and it will reduce the problem of omitted variables, we also employ the fixed effect model although DID regression captures a heterogeneity between control and treatment groups.\\
\quad As a baseline model, we specify the following model.
\begin{align}
ln(Y)_{it}=\alpha_0+\alpha_1Treatment_{i}+\bm{\alpha_2T_t}+\bm{\alpha_3}Treatment_i\times \bm{T_t}+\bm{X_{it}\beta}+\varepsilon_{it}
\end{align}
$Y$ corresponds to "the borrowing per capita+1". Here, we add 1 to the borrowing per capita because the borrowing tends to be 0 in many municipalities and the data will be omitted there if we take a natural logarithm. $Treatment_i$ takes 1 if municipality $i$ belongs to the treatment group. $\bm{T_t}$ consist of two dummy variables which take 1 if the first (or second) half year of FY2015/16 and takes 0 otherwise.  Although municipalities to be merged were announced in January 2016 for sure, since MDB published the final decision of demarcation in December 2015 and municipalities could reach the information, we include the first half year of FY2015/16 as a treatment period. Capturing the control variables in $\bm{X_{it}\beta}$, the effect of treatment will be shown in the coefficient of cross term, $\bm{\alpha_3}$. In the regression, population, GVA, Area, and the total expenditure take natural logarithm values. The borrowing also takes natural logarithm after adding 1 because the borrowing tends to be zero. The baseline model is a basic DID regression model. However, we also employ the fixed effect model to capture a heterogeneity among each municipalities.
\begin{align}
ln(Y)_{it}=\alpha_0+\bm{\alpha_2T_t}+\bm{\alpha_3}Treatment_i\times \bm{T_t}+\bm{X_{it}\beta}+\mu_i+\varepsilon_{it}
\end{align}
The fixed effect $\mu_i$ is introduced and the control variables contain no time-invariant variables here\footnote{In addition, since we consider that the data of the borrowing may be censored at zero, we employed the following tobit model for the analysis of the borrowing.
\begin{align}
y=&\begin{cases}
ln(Borrowing per person+1)_{it}\text{ \ \ if \ }ln(Y)_{it}>0\\
0\text{ \ \ if \ }ln(Y)_{it}\le0
\end{cases}\notag\\
&ln(Y)_{it}=\alpha_0+\bm{\alpha_2T_t}+\bm{\alpha_3}Treatment_i\times \bm{T_t}+\bm{X_{it}\beta}+\mu_i+\varepsilon_{it}\notag\\
&\varepsilon_{it}\sim N(0,\sigma)\notag
\end{align}
Since the result employed the fixed effect is more robust, dummies for each municipality is introduced in the regression equation for a latent variable. The result for this tobit model is shown in Appendix.}.\\
\quad Using the index of free-ride, the alternative models can also constructed as follows.
\begin{align}
ln(Y)_{it}&=\alpha_0+\alpha_4Index_i+\bm{\alpha_2T_t}+\bm{\alpha_5}Index_i\times \bm{T_t}+\bm{X_{it}\beta}+\varepsilon_{it}\label{fr1}\\
ln(Y)_{it}&=\alpha_0+\alpha_1Treatment_i+\bm{\alpha_2T_t}+\bm{\alpha_3}Treatment_i\times \bm{T_t}+\alpha_4Index_i+\bm{\alpha_5}Index_i\times \bm{T_t}+X_{it}\beta+\varepsilon_{it}\label{fr2}
\end{align}
$Index_i$ shows the free-ride index made by the data of GVA. $\bm{\alpha_5}$ captures the effect of the extent of free-ride. From the results of (\ref{fr1}) and (\ref{fr2}), we can see whether the amalgamation itself has an effect for the fiscal common pool, the extent of free-ride affects it, or both do. Introducing the fixed effect, additional models such as
\begin{align}
ln(Y)_{it}&=\alpha_0+\bm{\alpha_2T_t}+\bm{\alpha_5}Index_i\times \bm{T_t}+\bm{X_{it}\beta}+\mu_i+\varepsilon_{it}\\
ln(Y)_{it}&=\alpha_0+\bm{\alpha_2T_t}+\bm{\alpha_3}Treatment_i\times \bm{T_t}+\bm{\alpha_5}Index_i\times \bm{T_t}+\bm{X_{it}\beta}+\mu_i+\varepsilon_{it}
\end{align}
are also examined.
\section{Results}
\quad The results for the baseline model and the fixed effect model is shown in Table 2. Surprisingly, the coefficient of cross term for the second half year of FY2015/16 is negatively significant in the regression for the borrowing. This result is totally different from the previous research, where the most of papers show that the cross term of treatment dummy and treatment period is significantly positive and show that the fiscal common pool happens before the mergers. On the other hand, the result about the total expenditure shows that the announcement of mergers has no significant effect. We can also see that the cross term for the first half year of FY2015/16 is not significant. This result is quite reasonable since the announcement for mergers was not arrived in all municipalities in the first half year of FY2015/16. 
\begin{center}Table 2:Results for the baseline model and the fixed effect model\\
\scriptsize
\begin{tabular}{lcc} \hline
 & (1) OLS& (2)FE \\
VARIABLES & \multicolumn{2}{c}{$\log($Borrowing per capita)} \\ \hline
 &  &  \\
log(Pop) & -0.331*** & 0.555 \\
 & (0.0477) & (2.186) \\
log(GVA) & 0.490*** & 4.334* \\
 & (0.0463) & (2.203) \\
log(Area) & -0.0442 &  \\
 & (0.0386) &  \\
ANC seat & -1.507*** &  \\
 & (0.180) &  \\
Metro & 1.638*** &  \\
 & (0.285) &  \\
Secondary City & 1.013*** &  \\
 & (0.175) &  \\
Treatment & -0.140* &  \\
 & (0.0784) &  \\
Treatment$\times$ FY2015/16-F& -0.150 & -0.159 \\
 & (0.184) & (0.155) \\
Treatment$\times$ FY2015/16-S & -0.355* & -0.368** \\
 & (0.181) & (0.161) \\
Constant & 1.766*** & -40.42 \\
 & (0.611) & (27.28) \\ \hline
Time dummy &Yes  &Yes  \\
Observations & 2,317 & 2,317 \\
R-squared & 0.336 & 0.227 \\
 Number of code &  & 232 \\ \hline
\multicolumn{3}{c}{ Robust standard errors in parentheses} \\
\multicolumn{3}{c}{ *** p$<$0.01, ** p$<$0.05, * p$<$0.1} \\
\end{tabular}
\end{center}\normalsize
\quad The results of additional regressions for the borrowing, which contains the index of free-ride, are shown in Table 3. Unlike the dummy of treatment, the cross terms constructed by the index of free-ride do not take significant values if the treatment dummy is not included although it turns out positively significant if the treatment is considered. On the other hand, the cross terms of the treatment dummy remain significantly negative in Table 3. This may show that the results about the index of free-ride are not robust while the results about the treatment dummy are robust.\\
\quad However, the magnitudes of the cross term may be problematic. Table 2 shows that the magnitude of the cross term will be $-0.368 \sim -0.355$, which means the announcement of mergers reduce 30\% of the borrowing per capita. On the other hand, the magnitudes of the cross term in Table 3 are doubled. This may be affected by the existence of the free-ride index, although it does not necessarily take significant value.\\  
\begin{center}
Table 3 :Results for the borrowing per capita and the index of free-ride\\
\scriptsize
\begin{tabular}{lcccc} \hline
 & (1) OLS& (2) FE& (3) OLS& (4) FE\\
VARIABLES & \multicolumn{4}{c}{ log(Borrowing per capita)}\\ \hline
 &  &  &  &  \\
log(Pop) & -0.337*** & 0.467 & -0.337*** & 0.918 \\
 & (0.0476) & (2.202) & (0.0477) & (2.208) \\
log(GVA) & 0.471*** & 4.265* & 0.472*** & 4.363** \\
 & (0.0466) & (2.191) & (0.0472) & (2.205) \\
log(Area) & -0.0542 &  & -0.0542 &  \\
 & (0.0385) &  & (0.0385) &  \\
ANCseat & -1.504*** &  & -1.503*** &  \\
 & (0.180) &  & (0.180) &  \\
Metro & 1.682*** &  & 1.683*** &  \\
 & (0.285) &  & (0.286) &  \\
Secondary city & 1.021*** &  & 1.021*** &  \\
 & (0.175) &  & (0.175) &  \\
Free-ride & -0.387*** &  & -0.509** &  \\
 & (0.0967) &  & (0.213) &  \\
Free-ride$\times$ FY2015/16-F & 0.0590 & 0.0665 & 0.629 & 0.690 \\
 & (0.168) & (0.131) & (0.542) & (0.477) \\
Free-ride$\times$ FY2015/16-S & -0.130 & -0.121 & 0.798 & 0.877* \\
 & (0.184) & (0.154) & (0.497) & (0.454) \\
Treatment &  &  & 0.0990 &  \\
 &  &  & (0.152) &  \\
Treatment$\times$ FY2015/16-F &  &  & -0.458 & -0.498 \\
 &  &  & (0.419) & (0.364) \\
Treatment$\times$ FY2015/16-S &  &  & -0.746* & -0.800** \\
 &  &  & (0.387) & (0.350) \\
Constant & 2.079*** & -38.85 & 2.061*** & -44.84 \\
 & (0.630) & (27.41) & (0.628) & (27.60) \\ \hline
Time dummy &Yes  &Yes  &Yes  &Yes  \\
Observations & 2,317 & 2,317 & 2,317 & 2,317 \\
R-squared & 0.337 & 0.226 & 0.338 & 0.229 \\
 Number of code &  & 232 &  & 232 \\ \hline
\multicolumn{5}{c}{ Robust standard errors in parentheses} \\
\multicolumn{5}{c}{ *** p$<$0.01, ** p$<$0.05, * p$<$0.1} \\
\end{tabular}
\end{center}
\quad These results mean that municipalities reduced their debt issuance after they received the announcement of mergers although the relative size of municipalities did not affect the behavioral change. This result is very surprising because the result is totally opposite to the hypothesis of the fiscal common pool problem. Why such a mysterious behavioral change was observed? From the interviews for merged municipalities, we could obtain some possible explanations. First one is that municipalities to be merged were assembled to create a new municipality and  monitored each other after they received the announcement. The assembly was formed and led by the provincial government, where municipalities were expected to make no new contracts in order to arrange a good financial condition for the new municipality. Thus, municipalities could not borrow money during the preparation period for mergers even if they had borrowed before. While the strength of such regulations was said to be decided in each province and depended on case by case, the regulations seemed to be implemented in each municipality to be merged. Second possible explanation is that there was no need to raise a money before the mergers. Many staff in merged municipalities told that the most of administrative cost entailed by the mergers had been a few and had spent after a new municipality had formed. Therefore, they said that the expenditure after receiving the announcement was basically the same as before the announcement although new contracts and new recruitment were regulated then. Some one pointed out that if a municipality received the announcement had spent money for an irregular expense, it would have been seen as an useless expenditure. In addition, South African municipalities are obliged to submit the data about their revenue and expenditure monthly to the national treasury and it is published. Therefore, probably the monitoring was relatively easy.\\
\quad To check the robustness of this result, we implemented the regression analyses using the yearly data and the quarterly data. Both regressions show that the borrowing in treatment municipalities reduced after the announcement was received. The result employed a tobit model also shows a similar result. For a DID analysis, it is the parallel trend assumption that is very important. Pseudo test was done to check it, but pseudo variables were not significant. In addition, except the tobit model, all results were calculated using the robust standard errors. Thus, the result which shows the treatment group reduced the borrowing is robust.
\section{Conclusion}
\quad Employing South African data, this paper analyzed the fiscal common pool problem entailed by municipal mergers. Although many papers show that municipalities to be merged increase their debt issuance before their mergers, this paper shows that South African municipalities reduce their borrowing before their mergers. This result may be difficult to understand considering the previous literature. However, this result shows that, with the regulation to make no new contract and a good monitoring, the fiscal common pool problem can be prevented.\\
\quad This paper also shed the new light to the literature about the fiscal common pool problem in terms of expanding the field to the developing countries. More and more developing countries will need the reform of the local public finance and more research about developing countries are needed. This paper shows that the South African municipal demarcation system prevents the fiscal common pool problem and this result may be a good example for other countries. However, we also find that other problems such as poor fiscal viability with South African municipalities. It is true that the municipal demarcation is a tool for more integrated administrations. However, it does not solve all problems and more sophisticated policies are needed. In order to seek such policies, more research should be done in this field. 
\bibliographystyle{aer}
\bibliography{reference1}
\end{document}